\documentclass{beamer}
 
\usepackage[utf8]{inputenc}
\usepackage{textpos}

% Font configuration
\usepackage{fontspec}
\setmainfont{Nunito}
\setsansfont{Nunito Sans}
\setmonofont{Source Sans Pro}

% Minted configuration for source code highlighting
\usepackage{minted}

% Use the include theme
\usetheme{codecentric}

% Metadata
\title{Demo Beamer Template}
\author{Markus Hauck}
\date{\today}

% The presentation content
\begin{document}
 
\frame{\titlepage}

\section{Introduction}
\label{sec:introduction}
 
\begin{frame}
\frametitle{The First Awesome Slide}
\begin{itemize}
\item \textbf{text in bold}
\item \textit{text in italics}
\item \texttt{text in mono}
\end{itemize}
\end{frame}

\section{Main Part}
\label{sec:main-part}
\begin{frame}
  \frametitle{The Main Part}
  \begin{itemize}
  \item This is where
  \item the really interesting
  \item stuff happens!
  \end{itemize}
\end{frame}

\begin{frame}[fragile]
  \frametitle{I can haz Source Code}
  \begin{center}
\begin{minted}{scala}
object Main extends App {
  implicit val sys = ActorSystem()
  implicit val mat = ActorMaterializer()
  
  sys.actorOf[MainActor]
}
\end{minted}
  \end{center}
\end{frame}

\begin{frame}[fragile]
  \frametitle{I can haz Source Code more}
  \begin{center}
\begin{minted}{haskell}
main :: IO ()
main = shakeArgs shakeOptions $ do
  want ["world dominance"]

  "world dominance" %> \out ->
    executePlan out
\end{minted}
  \end{center}
\end{frame}

\section{Conclusion}
\label{sec:conclusion}

\begin{frame}
  \begin{center}
    \huge
    Your conclusion here
  \end{center}
\end{frame}
 
\end{document}

